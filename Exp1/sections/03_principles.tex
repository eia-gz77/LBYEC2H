\subsection{What are the necessary and relevant concepts, principles, theoretical and design considerations for understanding the coursework and for supporting the correct results?}
\label{sec:nrcp}
\begin{enumerate}
\item Mesh Analysis to derive the system of linear equations for each loop.
\item Matrix Algebra to model the system of linear equations including the solution methods: Cramer's Rule and Inverse Matrix.
\item Eigen Theory to understand that eigenvalues are roots of the characteristic polynomial.
\end{enumerate}

\subsection{How does any new component, not covered in previous coursework, function?}
\label{sec:nfxn}
By:
\begin{enumerate}
\item Allowing the definition of symbolic variables, the Symbolic Math Toolbox worked to solve equations algebraically rather than numerically.
\item Computing the eigenvalues and eigenvectors using the eig function.
\end{enumerate}
	
\subsection{What figures, equations, and/or tables could support your answers in Sec.~\ref{sec:nrcp} and Sec.~\ref{sec:nfxn}?}
\begin{enumerate}
\item Figure 1 shows the 4-mesh resistive circuit diagram with voltage sources and resistor values used to generate the equations.
\end{enumerate}
	
\subsection{Did you cite more than two publications in your answers in Sec.~\ref{sec:nrcp} and Sec.~\ref{sec:nfxn}?}
Yes.
\subsection{Did you cite any online source in your answers in Sec.~\ref{sec:nrcp} and Sec.~\ref{sec:nfxn}?}
No.