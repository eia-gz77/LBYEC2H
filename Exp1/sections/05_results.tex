\subsection{How do the results achieve the objectives?}
By:
\begin{enumerate}
\item Calculating the specific mesh currents $i_1$ through $i_4$ using Cramer's Rule, Matrix Inverse, and the \texttt{solve} function.
\item Determining the characteristic polynomial, eigenvalues, and eigenvectors of the circuit's coefficient matrix.
\end{enumerate}
    
\subsection{Why do the results achieve the objectives?}
Because:
\begin{enumerate}
\item The consistency of current values across three different calculation methods verifies the accuracy of the linear algebra implementation.
\item The extraction of eigenvalues demonstrates the successful application of spectral theory to the circuit matrix using MATLAB.
\end{enumerate}

\subsection{Are all your results correct in accordance with what you described in Sec.~\ref{sec:eval} evaluation process? Why?} 
Yes, because:
\begin{enumerate}
\item The output for current $i_1$ remained exactly $0.0556$ A regardless of whether Cramer's rule or Matrix Inverse was used.
\item The eigenvalues calculated manually via the characteristic polynomial matched the outputs of the built-in \texttt{eig(A)} function.
\end{enumerate}
    
\subsection{What is Result 1 (Mesh Currents), what does it mean if it is correct, and how does it contribute to reaching the objectives?}
\label{sec:res1}
        
        \begin{enumerate}       
        \item Table~\ref{tab:currents} (or the MATLAB output in the Appendix) displays the calculated loop currents: $i_1=0.0556$, $i_2=0.0338$, $i_3=0.0717$, and $i_4=0.1009$.   
        \item These positive values indicate that the actual current flow direction matches the assumed clockwise direction of the mesh loops.
        \item There were no discrepancies; the values satisfied the original KVL equations when substituted back.
        \begin{enumerate}
            \item The Matrix Inverse method returned identical floating-point values to the symbolic solution.
            \item Cramer's rule determinants yielded the same ratios.
        \end{enumerate}
        
        \item Standard circuit analysis textbooks confirm that unique solutions exist for linear resistive circuits with independent sources.
        \item This result confirms that the system of linear equations $\vec{A}\vec{i}=\vec{c}$ was correctly formulated and solved.
\end{enumerate}

\subsection{What is Result 2 (Eigenvalues), what does it mean if it is correct, and how does it contribute to reaching the objectives?}
\label{sec:res2}
        
        \begin{enumerate}       
        \item The eigenvalues of Matrix A were found to be $\lambda \approx 141.78, 115.27, -66.49, -146.56$.   
        \item These values represent the scalar factors by which the eigenvectors are scaled by the linear transformation of the circuit matrix.
        \item Discrepancies were negligible, limited only to minor floating-point variations in the 4th decimal place.
        \begin{enumerate}
            \item The characteristic polynomial calculation \texttt{roots(p)} aligned with the direct \texttt{eig(A)} function.
            \item Small numerical noise (e.g., -0.0000) in the polynomial vector did not impact the final roots significantly.
        \end{enumerate}
        
        \item This analysis is standard in linear algebra coursework for understanding matrix properties.
        \item This result validates the student's ability to manipulate matrices beyond simple system solving, covering spectral concepts.
\end{enumerate}
    
\subsection{Did you cite more than two publications in your answers above (yes/no)?}
No.