\subsection{How does your implementation in Sec.~\ref{sec:implem} achieve the objectives?}
By:
\begin{enumerate}
\item Up to two lines per item.
\item Up to two lines per item.
\end{enumerate}
	
\subsection{Why does your implementation in Sec.~\ref{sec:implem} achieve the objectives?}
Because:
\begin{enumerate}
\item Up to two lines per item.
\item Up to two lines per item.
\end{enumerate}
	
\subsection{How does your evaluation in Sec.~\ref{sec:eval} achieve the objectives?}
By:
\begin{enumerate}
\item Up to two lines per item.
\item Up to two lines per item.
\end{enumerate}
	
\subsection{Why does your evaluation in Sec.~\ref{sec:eval}  achieve the objectives?}
Because:
\begin{enumerate}
\item Up to two lines per item.
\item Up to two lines per item.
\end{enumerate}

\subsection{Implementation}
\label{sec:implem}

Rule of thumb: Implementation is how you made your work;
(keywords: implemented, created, made, soldered, programmed, etc.).


\subsubsection{What were the materials used?}
If the presentation would be better and necessary, tabulate your answers here instead of enumeration.
\begin{enumerate}
\item Up to two lines per item.
\item Up to two lines per item.
\end{enumerate}


\subsubsection{What is the summary of the processes used to make the coursework?}
If you wrote a program or made a simulation, you must add statements on how the program or simulation functions in this section.
A pseudocode as shown in Table~\ref{tab:calcxn} is a good example.
\begin{enumerate}
\item Up to two lines per item.
\item Up to two lines per item.
\end{enumerate}

\begin{table}[!b]
	\caption{Pseudocode for the calculation of $y = x^n$}
	\label{tab:calcxn}	
	\centering
	{\footnotesize
	\begin{tabular}{lll}
	\hline
	\hline
	{\bfseries Input(s):} & & \\
	$n$ & : & $n$th power;
$n \in \mathbb{Z}^{+}$ \\
	$x$ & : & base value; $x \in \mathbb{R}^{+}$ \\
	\hline
	{\bfseries Output(s):} & & \\
	$y$ & : & result;
$y \in \mathbb{R}^{+}$  \\
	\hline
	\hline
	\\
	\end{tabular}
	}
	\begin{algorithmic}[1]
	{\footnotesize
		\REQUIRE $n \geq 0 \vee x \neq 0$
		\ENSURE $y = x^n$
		\STATE $y \Leftarrow 1$
		\IF{$n < 0$}
				\STATE $X \Leftarrow 1 / x$
				\STATE $N \Leftarrow -n$
		\ELSE
				\STATE $X \Leftarrow x$
				\STATE $N \Leftarrow n$
		\ENDIF
		\WHILE{$N \neq 0$}
				\IF{$N$ is even}
						\STATE $X \Leftarrow X \times X$
						\STATE $N \Leftarrow N / 2$
				\ELSE[$N$ is odd]
						\STATE $y \Leftarrow y \times X$
						\STATE $N \Leftarrow N - 1$
				\ENDIF
		\ENDWHILE
	}
	\end{algorithmic}
\end{table}

\subsection{Evaluation}
\label{sec:eval}

Rule of thumb: Evaluation is how you tested your work for correctness;
(keywords: measured, tested, compared, simulated, etc.).

\subsubsection{What were your procedures for evaluating the correct outcome of your coursework?}
\begin{enumerate}
\item Up to two lines per item.
\item Up to two lines per item.
\end{enumerate}
	
\subsubsection{What quantities were gathered and how have you obtained them for testing the veracity of your results?}
\begin{enumerate}
\item Up to two lines per item.
\item Up to two lines per item.
\end{enumerate}