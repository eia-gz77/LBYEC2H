\subsection{How does your implementation in Sec.~\ref{sec:implem} achieve the objectives?}
By:
\begin{enumerate}
\item Formulating four simultaneous mesh equations based on the circuit diagram loops.
\item Converting the linear system into matrix form $\vec{A}\vec{i}=\vec{c}$ for computational processing.
\end{enumerate}
    
\subsection{Why does your implementation in Sec.~\ref{sec:implem} achieve the objectives?}
Because:
\begin{enumerate}
\item Matrix representation enables the application of algorithmic solutions like Cramer's Rule and Matrix Inverse.
\item The Symbolic Math Toolbox allows for exact algebraic manipulation of variables before numerical evaluation.
\end{enumerate}
    
\subsection{How does your evaluation in Sec.~\ref{sec:eval} achieve the objectives?}
By:
\begin{enumerate}
\item Solving the system using three distinct methods: Cramer's Rule, Matrix Inverse, and the built-in `solve` function.
\item Computing eigenvalues and eigenvectors using both characteristic polynomials and the standard `eig` function.
\end{enumerate}
    
\subsection{Why does your evaluation in Sec.~\ref{sec:eval}  achieve the objectives?}
Because:
\begin{enumerate}
\item Comparing results across multiple calculation methods ensures the numerical accuracy of the calculated currents.
\item Verifying manual algorithmic logic (loops/determinants) against optimized built-in functions validates the code structure.
\end{enumerate}

\subsection{Implementation}
\label{sec:implem}

Rule of thumb: Implementation is how you made your work;
(keywords: implemented, created, made, soldered, programmed, etc.).


\subsubsection{What were the materials used?}
\begin{enumerate}
\item Personal Computer with MATLAB software installed.
\item Laboratory exercise manual and circuit diagram for mesh analysis.
\end{enumerate}


\subsubsection{What is the summary of the processes used to make the coursework?}
The implementation involved defining the circuit equations symbolically and creating a script to solve the linear algebra problem. A pseudocode for the general solving process is shown in Table~\ref{tab:linsolve}.
\begin{enumerate}
\item Defined symbolic variables $i_1, i_2, i_3, i_4$ and wrote mesh equations based on KVL.
\item Constructed the coefficient matrix $A$ and constant vector $c$ from the linear equations.
\end{enumerate}

\begin{table}[!b]
    \caption{Pseudocode for Solving System of Linear Equations}
    \label{tab:linsolve}  
    \centering
    {\footnotesize
    \begin{tabular}{lll}
    \hline
    \hline
    {\bfseries Input(s):} & & \\
    $A$ & : & Coefficient Matrix ($4 \times 4$) \\
    $c$ & : & Constant Vector ($4 \times 1$) \\
    \hline
    {\bfseries Output(s):} & & \\
    $i$ & : & Loop Currents vector ($4 \times 1$) \\
    \hline
    \hline
    \\
    \end{tabular}
    }
    \begin{algorithmic}[1]
    {\footnotesize
        \REQUIRE $\det(A) \neq 0$
        \ENSURE $A \cdot i = c$
        \STATE Define symbolic variables $eqn1 \dots eqn4$
        \STATE $A \Leftarrow$ Extract Coefficients from equations
        \STATE $c \Leftarrow$ Extract Constants from equations
        \STATE Calculate Determinant $D \Leftarrow \det(A)$
        \IF{Method is Cramer's Rule}
             \FOR{$k = 1$ to $4$}
                \STATE $A_k \Leftarrow A$
                \STATE Replace $k$-th column of $A_k$ with $c$
                \STATE $i_k \Leftarrow \det(A_k) / D$
             \ENDFOR
        \ELSIF{Method is Matrix Inverse}
             \STATE $A_{inv} \Leftarrow \text{inv}(A)$
             \STATE $i \Leftarrow A_{inv} \times c$
        \ENDIF
        \STATE Display $i$
    }
    \end{algorithmic}
\end{table}

\subsection{Evaluation}
\label{sec:eval}

Rule of thumb: Evaluation is how you tested your work for correctness;
(keywords: measured, tested, compared, simulated, etc.).

\subsubsection{What were your procedures for evaluating the correct outcome of your coursework?}
\begin{enumerate}
\item The calculated currents from Cramer's Rule were cross-referenced with the Matrix Inverse method results.
\item Eigenvalues derived from the `poly` and `roots` functions were checked against the `eig` function output.
\end{enumerate}
    
\subsubsection{What quantities were gathered and how have you obtained them for testing the veracity of your results?}
\begin{enumerate}
\item Loop currents ($i_1 \approx 0.0556$, $i_2 \approx 0.0338$, etc.) obtained via determinants and inverses.
\item Eigenvalues (e.g., $141.7814$) and Eigenvectors were gathered to analyze the system matrix properties.
\end{enumerate}